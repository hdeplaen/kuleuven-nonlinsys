\chapter{Study of a predator-prey model}

\begin{center}
\colorbox{LightBlue}{\parbox{1\textwidth}{\noindent{\bf Note:} There is no need to repeat the information provided during the guided session. Answer the questions as concise and accurate as possible. Use graphical and/or tabular presentation rather than full sentences. The \textit{page limit} 
for this exercise is \textit{6 pages text}, \underline{not} including figures.}}
\end{center}

We consider a predator-prey model
\begin{equation*}
\left\{
    \begin{aligned}
        \dot{x} &= x (x - a) (1 - x) - b x y, \\
        \dot{y} &= xy - cy - d,
    \end{aligned}
\right.
\end{equation*}

in which $a, b, c$ and $d$ are physical parameters.

\begin{center}
\colorbox{LightBlue}{\parbox{1\textwidth}{\noindent The values of the para\-meters $a$ and $b$ are fixed and given to you (see \texttt{parameters.pdf} on Toledo). Try always to stay symbolical as long as possible. The numerical values are meant for the plots and the extreme cases where an analytical solution is very hard to find.}}
\end{center}

\noindent Give a brief ecological interpretation to each term and parameter in the equations. You do not have to continue this interpretation later on in the assignment.

\begin{Exercise}[name={A qualitative study without external influence}]\label{EX31}

We consider the parameter $d=0$, and confine ourselves to
parameter values $c \in [0,1.5]$.


\Question First consider the case $y=0$.  Perform a numerical
simulation for a number of initial values $x_0 \in [0,1.5]$ and
describe briefly the qualitative behaviour of the system as a function of time.
\Question For the two-dimensional problem, compute analytically the steady state solutions and determine for each steady state its stability and the topological structure of the
phase diagram in the neighbourhood. For the equilibria in which
$y=0$, give the slow eigendirection of the stable and unstable nodes, and determine analytically the stable and the unstable manifolds for the saddles. \rema{Report your findings in a table and/or some figures.}

\textbf{Hint:} Use the trace and determinant of the Jacobian matrix \pj{or compute its the eigenvalues} to find the topological structure around an equilibrium as a function of the parameter $c$. 

\textbf{Example:}
For the equilibrium $(c,(c - a)(1-c)/b)$, the Jacobian matrix is given by
\begin{equation} 
J=\left[ \begin{array}{cc} 
c(1+a-2c) &  -b c \\
(c-a)(1-c)/b &  0 
\end{array} \right] \nonumber
\end{equation}
The trace $\tau = c(1+a-2c)$ is zero if $c=0$ or $c = (1+a)/2$. The determinant $\Delta = c(c-a)(1-c)$ is zero if $c=0$, $c=a$ or $c=1$. 

\Question Confirm your analysis by drawing phase diagrams for some (relevant) values of $c$. 
Make sure your results are consistent with your analysis in item 2. What is the relation between (stable and unstable) manifolds
of saddle points and separatrices? How can you find the separatrices using a
combination of analytical and numerical techniques?  Determine the
regions of attraction of the attractors. \rema{Answer with some figures.}

\Question Use the results of the previous questions to give a
qualitative overview of the changes in the phase diagram when $c$
varies in $[0, 1.5]$. Give a qualitative picture of the evolution
of the eigenvalues of the Jacobian in the complex plane for the
steady state $(c,(1-c)(c-a)/b)$ as a function of the parameter $c$ and indicate which bifurcations occur.  

\Question Given the interval $[c_1, c_2]$ (see Toledo): between $c_1$ and $c_2$, an important global bifurcation
phenomenon occurs.  Monitor how the periodic solutions change as
the parameter $c$ changes; also look at the period.  Which
bifurcation occurs? At which value of $c$? Can you confirm the bifurcation with a scaling law for the period and/or amplitude of the limit cycles close to the bifurcation?  Draw the phase diagram
for the critical value of $c$.

\Question Compare the evolution of $x$ and $y$ as a function of time
for
\begin{tasks}
\task a limit cycle corresponding to an almost harmonic
oscillation;
\task a limit cycle close to a heteroclinic\footnotemark{} cycle.
\end{tasks}
\rema{Answer with some figures and briefly discuss the qualitative difference between these two
orbits.}

\end{Exercise}\footnotetext{A \emph{heteroclinic} cycle is similar to a \emph{homoclinic} one, at the exception of having multiple centers.}

\begin{Exercise}[name=Bifurcation analysis]

For this part of the assignment, we confine ourselves to $c \in
[0.1, 1.5]$. \textbf{Note:} although this assignment is shorter in text than 3.1, it is equally important (and time-consuming)!

\Question Draw the bifurcation diagram for $d=0$. 
Consider the $(x,c)$ projection.
Discuss the branches, and discuss how the stability changes at all
relevant bifurcation points.  Also compute the periodic solutions.
Make sure that your bifurcation diagram is consistent with the
phase diagrams and analysis of the first part of the assignment.

\Question Draw the bifurcation diagrams for $d=0.01$ and $d=-0.01$; 
Consider both $(x,c)$ and $(y,c)$ projections. 
Compare the results with the bifurcation diagram for $d=0$, and indicate how the transcritical bifurcations disappear. It might be useful to extend your figures to include 
negative values of $x$, $y$ and $c$ to see all branches.

\end{Exercise}
